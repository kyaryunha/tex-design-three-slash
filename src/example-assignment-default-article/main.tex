% 보고서 쓸 때 유용한 라이브러리들 
\documentclass{article}
\usepackage[utf8]{inputenc}
\usepackage{url}
\usepackage{graphicx}
\usepackage{geometry}
\usepackage{kotex}
\usepackage[dvipsnames]{xcolor}
\usepackage{tikz}
\usepackage{minted}
\usepackage{fontspec}
\usepackage{tcolorbox}
\usepackage{hyperref}
\usepackage[english]{babel}

% 색깔 설정들 
\definecolor{one}{RGB}{0 68 131}
\definecolor{two}{RGB}{150 150 150}
\definecolor{three}{RGB}{0 0 0}
\definecolor{graytwo}{RGB}{60 60 60}
\definecolor{graythree}{RGB}{80 80 80}


% TODO 입력하기 ("TODO" text with red color)
\newcommand{\TODO}[0]{{\color{red}\textbf{TODO}}}

% 글꼴 설정 
\setmainfont[Color=graytwo]{Roboto-Light}
\setmainhangulfont[Color=graythree]{Noto Sans CJK KR}

% 용지 모양 설정
\geometry{a4paper, left=25mm, right=15mm, top=20mm, bottom=0mm}
 
%  Title 설정 
\title{\includegraphics[height=4cm]{img/hanyang.png}
\\~\\ Assignment Example \#1
}
\date{May 16, 2020}
\author{한양대학교 컴퓨터소프트웨어학부\\2019024439 신 현}

% 코드 입력 단순화 ( cpp, python 등 다양한 언어 바꾸면 된다 )
\newcommand{\Code}[1]{
	\inputminted[]{cpp}{#1}
}

% nextpage... 페이지 넘김 직접해줘야 
% 이 문서의 핵심(?)인 왼쪽 디자인이 안깨진다 
% 언젠가 자동화해서 다시 푸시할게요... 
\newcommand{\nextpage}[0]{
    \newpage
    \begin{tikzpicture}[remember picture,overlay]
        \begin{scope}[every shade/.append style={yslant=0.5},yslant=0.5]
            \shade[top color=one,bottom color=one] (-4,4) rectangle (-2,-6.45);
            \shade[top color=two,bottom color=two] (-4,-6.45) rectangle (-2,-16.65);
            \shade[top color=three,bottom color=three] (-4,-16.65) rectangle (-2,-26.5);
        \end{scope}
    \end{tikzpicture}
}

% 첫 페이지에 쓰면 아래에 출처 표시해줌
\newcommand{\threeslash}{
    \begin{tikzpicture}[remember picture,overlay]
        \node[draw=none] at (14,-16) {github.com/kyaryunha/teX-design-three-slash};
    \end{tikzpicture}
}


\begin{document}
% 페이지 넘버링 설정 
% \pagenumbering{gobble}
\nextpage
% 장평 설정 
\setlength{\baselineskip}{15pt}
% maketitle 
{\let\newpage\relax\maketitle}

% 적당히 예쁜 위치에 목차 그리고 ( 보고서 양에 따라 생략 )
~\\~\\
\setlength{\baselineskip}{22pt}
\tableofcontents
\setlength{\baselineskip}{15pt}

% 페이지 다시 넘기고
\nextpage
% 보고서 내용 적으면 됨 

% Chapter와 Section의 이름을 의도적으로 변경하고, 이를 목차에 추가하거나 삭제하고 싶으면 아래와 같이 하자. ( 학교 보고서에는 Chapter와 Section 형식만을 써야하는 경우는 적으니.. ) 
\addcontentsline{toc}{section}{Part 1.~~C언어 출력하기}
\section*{Part 1. C언어 출력하기 }

\begin{tcolorbox}[title={Part 1. C언어 출력하기},
fonttitle={\sffamily\bfseries}]
	~\\
	Q1: Hello World를 출력해보자  \\
		
	Q2: I am Shin Hyun을 출력해보자
	
\end{tcolorbox}


\addcontentsline{toc}{subsection}{1.1~~~Hello World !}
\subsection*{1.1~~~Hello World를 출력해보자 }
\subsubsection*{코드}
\begin{tcolorbox}[colback=white!5!white, colframe=white!75!black]
    \Code{source/helloworld.cpp}
\end{tcolorbox}


\addcontentsline{toc}{subsection}{1.2~~~My name is Shin Hyun !!}
\subsection*{1.2~~~My name is Shin Hyun을 출력해보자 }
\subsubsection*{코드}
\begin{tcolorbox}[colback=white!5!white, colframe=white!75!black]
    \Code{source/shinhyun.cpp}
\end{tcolorbox}

\nextpage
\addcontentsline{toc}{section}{Part 2.~~C언어 입력받기}
\section*{Part 2. C언어 입력받기 }

\addcontentsline{toc}{subsection}{2.1~~~정수 입력 받기}
\subsection*{2.1~~~정수 입력 받기 }
내용은 생략하겠다. 

\addcontentsline{toc}{subsection}{2.2~~~문자 입력 받기}
\subsection*{2.2~~~문자 입력 받기 }
내용은 생략하겠다. 

\nextpage
\addcontentsline{toc}{section}{부록}
\subsection*{부록}
부록은 다음과 같다.
\begin{tcolorbox}[colback=white!5!white, colframe=white!75!black]
    \Code{source/Appendix.cpp}
\end{tcolorbox}



\end{document}